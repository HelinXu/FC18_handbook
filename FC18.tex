%%% Reusable Start
\documentclass[a4paper,4pt]{article}
\usepackage[UTF8]{ctex}
\usepackage{geometry}
\usepackage{titlesec}
\usepackage{indentfirst}
\usepackage{graphicx} %use graph format
\usepackage{subfigure}
\usepackage{epstopdf}
\usepackage{hyperref}
\usepackage{float}
\usepackage{amsmath}
\usepackage{amssymb}
\usepackage{multirow}
\usepackage{array}
\usepackage{enumerate}
\usepackage{enumitem} 
\usepackage{bookmark}
\usepackage{booktabs}
\usepackage{multirow}
\usepackage{bigstrut}
\usepackage[backend=bibtex]{biblatex}
\usepackage{pgfplots}
\usepackage{pgfplotstable}
\usepackage{siunitx}
\usepackage{datetime}
%插入中文日期
\renewcommand{\today}{\number\year 年 \number\month 月 \number\day 日}
%% for inserting code
\usepackage{listings}
\usepackage{xcolor}
\definecolor{mygreen}{rgb}{0.3,0.7,0}  
\definecolor{mygray}{rgb}{0.5,0.5,0.5}  
\definecolor{myblue}{rgb}{0.4,0.7,1}
\definecolor{mywhite}{rgb}{0.9,0.9,0.9}
\definecolor{mymauve}{rgb}{0.58,0,0.82}
\definecolor{mybackcolor}{RGB}{62,62,62}
\definecolor{myIdt}{rgb}{0.8,0.6,1}
\lstset{ %  
  backgroundcolor=\color{mybackcolor},   % choose the background color; you must add \usepackage{color} or \usepackage{xcolor}  
  basicstyle=\small\ttfamily\color{mywhite},        % the size of the fonts that are used for the code  
  breakatwhitespace=false,         % sets if automatic breaks should only happen at whitespace  
	breaklines=true,                 % sets automatic line breaking  
  captionpos=bl,                    % sets the caption-position to bottom  
  commentstyle=\color{mygreen},    % comment style  
  %deletekeywords={...},            % if you want to delete keywords from the given language  
  escapeinside={\%*}{*)},          % if you want to add LaTeX within your code  
  extendedchars=true,              % lets you use non-ASCII characters; for 8-bits encodings only, does not work with UTF-8  
  %frame=single,                    % adds a frame around the code  
  keepspaces=true,                 % keeps spaces in text, useful for keeping indentation of code (possibly needs columns=flexible)  
  keywordstyle=\color{myblue},       % keyword style  
  %language=Python,                 % the language of the code  
  morekeywords={*,...},            % if you want to add more keywords to the set  
  numbers=left,                    % where to put the line-numbers; possible values are (none, left, right)  
  numbersep=-15pt,                   % how far the line-numbers are from the code  
  numberstyle=\small\color{mygray}\ttfamily, % the style that is used for the line-numbers  
  rulecolor=\color{black},         % if not set, the frame-color may be changed on line-breaks within not-black text (e.g. comments (green here))  
  showspaces=false,                % show spaces everywhere adding particular underscores; it overrides 'showstringspaces'  
  showstringspaces=false,          % underline spaces within strings only  
  showtabs=false,                  % show tabs within strings adding particular underscores  
  stepnumber=1,                    % the step between two line-numbers. If it's 1, each line will be numbered  
  stringstyle=\color{orange},     % string literal style  
  tabsize=2,                       % sets default tabsize to 2 spaces  
  %title=myPython.py                   % show the filename of files included with \lstinputlisting; also try caption instead of title  
	rulesepcolor=\color{red!20!green!20!blue!20},
	identifierstyle=\color{myIdt},
	} 
%% end of code style



\begin{document}

%第二种封面
\begin{titlepage}
	\heiti
	\vspace*{64pt}
	\begin{center}
		\fontsize{43pt}{0} \textbf{FC18参赛规则}\\
		\vspace*{36pt}
		\fontsize{43pt}{0}{内测版}\\
		\vspace*{280pt}
		\Large
		\rmfamily
		\begin{tabular}{lr} %表格,lr表示左对齐|右对齐
			\textbf{开发者}   & 自动化系科协竞赛部 \\
			\textbf{比赛网址} & ?                 \\
			\textbf{?}       & ?                 \\
			\textbf{更新日期} & \today             \\
		\end{tabular}
	\end{center}
\end{titlepage}


\tableofcontents%目录
\newpage%换页

FC18为四方势力在地图上进行回合制对战的策略游戏,玩家需要力求攻占其他玩家的塔、占领尽可能大的领地、消灭或俘虏其他玩家的兵团以获得胜利。每个玩家需要编写AI,根据裁判程序提供的场地信息,决策己方势力在该回合的行动,并返回给裁判程序,以控制己方的行为。
\section{游戏说明}
作为塔防游戏,每个势力在开始时各在一角拥有一座塔,塔周围的一定区域是自己的领地。玩家需要利用塔的生产力,完成生产兵团或升级塔的任务。生产出的作战兵团可以在场上移动,攻击其他势力的塔或兵团(减少他们的生命值);生产出的工程兵团则可以完成修改地形、修理塔(恢复塔的生命值)的任务。其中,塔的等级越高,就会拥有更高的生产力、战斗力、生命值上限。塔也可以攻击敌方的兵团,如果塔内有己方兵团驻扎,则塔的战斗力会更强。

\begin{figure}[htbp]   %用htbp表示插入位置自动排版
	\centering
	\includegraphics[width=5.5 in]{01开始.jpg}
	\caption{游戏场地示意图:开始状态}
	\label{jpg:示例图片1}
\end{figure}

\section{你需要做什么}
所有玩家的AI都可以从Info中读取当前场上各方势力的兵团、塔的信息,并设计算法,并按照统一的接口CommandList给裁判程序回传命令,操控己方势力;在游戏中,选手只需要在ai.cpp文件中的void player\_ai(Info\& info) {  }函数中填写自己的代码,并最终只需要提交ai.cpp文件。
\begin{lstlisting}[language={C++},title={添加命令有关代码}]  %插入代码块
    //【FC18】命令列表
    class CommandList
    {
    public:
      void addCommand(commandType _FC18type, initializer_list<int> _FC18parameters);
      void removeCommand(int n);  //【FC18】移除第n条命令
      vector<Command> getCommand() { return m_commands; }  //【FC18】获取所有命令
    };
\end{lstlisting}

\section{角色说明}
游戏中有以下角色:塔$Tower$,作战兵团(分为战士$Warrior$、弓箭手$Archer$、法师),工程兵团(分为建设者$Builder$、开拓者$Extender$)。
\subsection{塔}
\subsubsection{概述}
塔是在地图上固定的,具有对一定范围内(距离为2及以内,即5*5的方形内)敌对势力塔或兵团自动攻击能力的建筑,每个势力最初时拥有1座塔,每个地图方格内最多有1座塔,每个势力最多拥有10座塔。当某势力没有塔时,该势力将被判为失败。
\begin{figure}[htbp]   %用htbp表示插入位置自动排版
	\centering
	\includegraphics[width=1.5 in]{距离.jpg.png}
	\caption{距离计算示意图}
	\label{jpg:示例图片2}
\end{figure}
塔有以下属性:
\begin{enumerate}[fullwidth, itemindent=2em, label=(\arabic*)]
	\item 等级$N$。
	\item 生产力$W_N$。
	\item 等级生命力上限$HP_N$,当前生命力$hp$。
	\item 等级战斗力$F_N$,实际战斗力$f$。%TODO
	      \begin{lstlisting}[language={C++},title={防御塔结构体}]  %插入代码块
  	//@@@【FC18】防御塔结构体,有需要的信息再加
  	
  	struct TowerInfo {
  		
  		TTowerID      ID;   //防御塔ID
  		TPlayerID     ownerID;  //所属玩家ID
  		TPoint        position;    //位置
  		TProductPoint productPoint;  //生产力
  		TProductPoint productConsume;  //当前生产任务尚需完成的剩余生产力值
  		TBattlePoint  battlePoint;   //战斗力
  		THealthPoint  healthPoint;   //生命值
  		TLevel        level;       //等级
  		productType   pdtType;    //当前生产任务类型
  	};
  \end{lstlisting}

\end{enumerate}
\subsubsection{具体说明}
1)塔的等级$N$为1-8的正整数,从等级$N$升级到$N+1$需要消耗的生产力值为$40 \cdot N$。\par
2)塔会根据自身的等级状况获得一个生产力数值$W_N$,代表其一个回合能产生的生产力,具体的数据如表\ref{塔等级表}所示。生产力$W_N$每回合更新,如果该回合没有使用生产力,下一回合不会累积。在同一时刻,塔可以选择一种生产任务(包括生产战士、生产弓箭手、生产法师、生产建造者、生产开拓者、升级塔自身)每一种任务需要消耗不同的生产力点数,具体情况见表\ref{塔生产}。兵团生产任务完成之后,塔所在的方格将立即生成一个对应兵团。防御塔所在方格内允许存在多个兵团,不必遵循同一单元格仅能存在一个兵团的限制。在struct\ TowerInfo中,可以访问到该塔当前未完成的生产任务productType\ pdtType;和该任务有待完成的工作余量TProductPoint\ productConsume;\par
3)等级生命力上限$HP_N$仅与等级正相关,具体的数据如表\ref{塔等级表}所示。实际生命力$hp$会受到进攻而减小(具体结算方式如式\ref{hp1}和式\ref{hp2}所示)、由于建设者的维护而增加。当实际生命力$hp$被进攻方降低至0以下时,对方获得5分的击杀分,我方丧失该塔,塔等级下降4级:等级下降后,若不足1级,则塔消失,该单位恢复原来地形;若塔尚存在,则降级后的塔主权归对方所有(塔被俘虏)。\par
4)实际战斗力$f$为战斗中的战斗力。$f$是$F_N$、当前生命值$hp$、等级生命值上限$HP_N$、兵团驻扎情况带来战斗力增益$f_c$(具体增益规则如表\ref{兵团}所示)的函数。具体计算方法如下:

\begin{equation}
	f = F_N \cdot \frac{hp}{HP_N} + \Sigma f_b\label{f}
\end{equation}
兵团攻击塔时,生命值的结算方式如下:
\begin{equation}
	\begin{aligned}
		hp_{\text{塔}}   & \ -= & \ 30 \cdot k_c \cdot e^{0.04(f_{\text{兵}}-f_{\text{塔}})} & ;                       \\
		hp_{\text{兵团}} & \ -= & \ 28 \cdot e^{0.04(f_{\text{塔}}-f_{\text{兵}})}           & , \text{当兵团非弓箭手} \\
		hp_{\text{兵团}} & \ -= & \ 0                                                        & , \text{当兵团为弓箭手} \\
	\end{aligned}
	\label{hp1}
\end{equation}
塔只能攻击距离为2及以内的兵团。塔攻击兵团时,生命值的结算方式如下:
\begin{equation}
	\begin{aligned}
		hp_{\text{兵团}} & \ -=\ 30 \cdot e^{0.04(f_{\text{塔}}-f_{\text{兵}})}, \text{对于所有兵团种类} \\
		hp_{\text{塔}}   & \ -=\ 0                                                                       \\
	\end{aligned}
	\label{hp2}
\end{equation}




你可以为塔添加以下操作:
\begin{enumerate}[fullwidth, itemindent=2em, label=(\arabic*)]
	\item 生产:在命令中指定命令类型为塔命令,塔操作类型为生产,塔ID,塔生产任务类型( <enum  productType> )。
	\item 攻击:在命令中指定命令类型为塔类型,塔操作类型为攻击,塔ID,塔的攻击对象序号,即可指定某一做塔攻击某一个对象。
\end{enumerate}
另外,请注意:
\begin{enumerate}[fullwidth, itemindent=2em, label=(\arabic*)]
	\item 每回合,每座塔最多可以添加一个操作。如果该回合玩家给一个塔添加了多个操作,则第二个及以后的操作会被自动忽略,且占用50个操作的余额(相当于填了废志愿),所以请不要添加大于一个操作。
	\item 如果你的操作是无效操作,则也不会起任何作用。无效操作包括试图攻击自己的对象、试图攻击超出攻击范围的对象、不存在对应序号的对象等。
	\item 如果上一回合塔选择了生产,但未完成该生产任务,本回合选择攻击,则上一回合的生产进度会被保留,这样在下一回合假如继续添加生产该任务的命令,会在上一回合基础上继续生产。
	\item 若防御塔的某个生产任务尚未完成,玩家又指定了新任务,则未完成的生产任务的完成度将被缓存起来,然后进行新的任务。之后再选择有一定完成度的任务的时候,只需要完成未完成的部分,即完成剩余的生产力消耗值。即如果上一回合塔选择了生产任务A,但未完成该生产任务A,本回合选择生产任务B,则上一回合的生产进度也会被保留,这样在下一回合假如继续添加生产任务A的命令,会在上一回合基础上继续生产。但是,由于接口所限,你只能通过TowerInfo访问到最近一次尚未完成的任务类型和余量,所以我们推荐一旦开始一种生产任务,中途不要切换别的生产任务(虽然你仍可以正常地生产它们)。
\end{enumerate}
\subsection{作战兵团}
在同一个时间,同一个地图方格(防御塔所在方格除外)内的作战兵团、工程兵团数量各自不能超过1个。当某个方格内同时存在一个势力的一个作战兵团和工程兵团,则称工程兵团被作战兵团护卫。任何战斗,都优先与作战兵团结算。若在某次战斗中,该护卫队中的作战兵团阵亡。此时进行一次判定,若敌方作战兵团也在作战中阵亡,则护卫队中的工程兵团仍属于原来的玩家;若敌方作战兵团未阵亡,则护卫队中的工程兵团将被敌方俘虏,所属玩家变更为敌方。\par
\subsubsection{概述}
兵团有三个种类:战士(近战单位)、弓箭手(远程单位)、法师(高级进攻单位)。
其中,弓箭手可以远程轰炸其他单位且自身不受伤害。法师具有较高的移动力。\par
作战兵团有以下属性:
\begin{enumerate}[fullwidth, itemindent=2em, label=(\arabic*)]
	\item 行动力$M_c$。
	\item 满血战斗力$F_c$,实际战斗力$f$。
	\item 生命力上限$HP_c$,当前生命力$hp$。
	\item 攻击距离$d_c$。%TODO
	\item 所属玩家ID。
\end{enumerate}
\begin{lstlisting}[language={C++},title={防御塔结构体}]  %插入代码块
		//@@@【FC18】兵团结构体,有需要的信息再加
		struct CorpsInfo
		{
			//不需要,如果不存在就不录入信息了bool	exist;		//是否存在
			TPoint	pos;		//兵团坐标
			int		level;		//兵团等级
			TCorpsID		ID;	//兵团ID
			THealthPoint	HealthPoint;	//生命值
			TBuildPoint		BuildPoint;		//劳动力
			TPlayerID		owner;			//所属玩家ID
			corpsType       type;           //兵团种类
			TMovePoint      movePoint;      //行动力
			battleCorpsType		m_BattleType;	//战斗兵用
			constructCorpsType	m_BuildType;	//建造兵用
		};
	\end{lstlisting}
\subsubsection{具体说明}
1)行动力$M_c$表示兵团在某一回合的进行行动的能力,具体见表\ref{兵团}。兵团从一个单元格移动到另一个相邻单元格(即上下左右四个方向)将消耗一定行动力。(取决于单元格地形情况,计算方式为:一次移动经过的两个单元格的行动力消耗取平均值,并向上取整,具体见表\ref{地形})值得注意的是,如果移动后行动力至少还有1点则则还能发起进攻,而一旦选择进攻将消耗所有行动力。行动力在新回合开始将重置。\par
2)生命力上限$HP_c$仅与兵团种类有关,具体见表\ref{兵团}。实际生命力$hp$会受到进攻而减小(具体结算方式如式\ref{hp2}所示),直至实际生命力$hp$被进攻方降低至0以下时,兵团死亡,对方获得5分的击杀分。\par
3)实际战斗力$f$为战斗中的战斗力。$f$是满血战斗力$F_c$、当前生命值$hp$、生命值上限$HP_c$、兵团所处地形情况带来战斗力增益$f_t$(具体增益规则如表\ref{地形}所示)的函数。具体计算方法如下:

\begin{equation}
	f = F_c \cdot \frac{hp}{HP_c} + f_t\label{f2}
\end{equation}
兵团只能攻击在攻击范围内的对象,不同兵团的攻击距离$d_c$如表\ref{兵团}所示。兵团B受到兵团A攻击时,生命值的结算方式如下:
\begin{equation}
	\begin{aligned}
		hp_{B} & \ -= & \ 30 \cdot e^{0.04(f_{A}-f_{B})} & ;                        \\
		hp_{A} & \ -= & \ 28 \cdot e^{0.04(f_{B}-f_{A})} & , \text{当兵团A非弓箭手} \\
		hp_{A} & \ -= & \ 0                              & , \text{当兵团A为弓箭手} \\
	\end{aligned}
	\label{hp2}
\end{equation}
作战兵团能进行的操作有:在地图上移动、停在某一方格内驻扎、驻扎己方势力的塔、对敌方军团发起进攻、对敌方防御塔发起进攻。
去掉兵团整编、兵团原地驻扎的命令(兵团驻扎到塔的指令还保留着)@TODO%TODO
\subsection{工程兵团}
工程兵团分为:建造者、开拓者。建造者用来进行特定的工程建造,而开拓者则用于修建新的防御塔。\par
工程兵团有以下属性:
\begin{enumerate}[fullwidth, itemindent=2em, label=(\arabic*)]
	\item 行动力$M_c$。
	\item 劳动力$B$。
	\item 所属玩家ID。
\end{enumerate}

%枚举


\begin{table}[htbp]
	\centering
	\caption{塔等级表}
	\label{塔等级表}%
	\begin{tabular}{c|c|c|c}
		\hline
		等级$N$ & 生产力$W_N$ & 等级战斗力$F_N$ & 等级生命力上限$HP_N$ \bigstrut \\
		\hline
		1       & 10          & 25              & 100 \bigstrut                  \\
		\hline
		2       & 15          & 27              & 120 \bigstrut                  \\
		\hline
		3       & 20          & 29              & 140 \bigstrut                  \\
		\hline
		4       & 25          & 32              & 160 \bigstrut                  \\
		\hline
		5       & 30          & 35              & 180 \bigstrut                  \\
		\hline
		6       & 35          & 38              & 200 \bigstrut                  \\
		\hline
		7       & 40          & 41              & 220 \bigstrut                  \\
		\hline
		8       & 45          & 45              & 240 \bigstrut                  \\
		\hline
	\end{tabular}%

\end{table}%




% Table generated by Excel2LaTeX from sheet 'Sheet1'
\begin{table}[htbp]
	\centering
	\caption{塔生产任务表}
	\begin{tabular}{c|c}
		\hline
		生产任务           & 所需的生产力值 \bigstrut \\
		\hline
		战士               & 40 \bigstrut             \\
		\hline
		弓箭手             & 60 \bigstrut             \\
		\hline
		法师               & 100 \bigstrut            \\
		\hline
		建造者             & 40 \bigstrut             \\
		\hline
		开拓者             & 40 \bigstrut             \\
		\hline
		升级(N升级到N+1) & N*40 \bigstrut           \\
		\hline
	\end{tabular}%
	\label{塔生产}%
\end{table}%

% Table generated by Excel2LaTeX from sheet 'Sheet1'
\begin{table}[htbp]
	\centering
	\caption{兵团参数表}
	\begin{tabular}{c|c|c|c}
		\hline
		兵种Crops           & 战士 & 弓箭手 & 法师 \bigstrut \\
		\hline
		战斗力增益系数$f_c$ & 2    & 2      & 4 \bigstrut    \\
		\hline
		攻城系数$k_c$       & 0.4  & 0.7    & 0.5 \bigstrut  \\
		\hline
		攻击距离$d_c$       & 1    & 2      & 1 \bigstrut    \\
		\hline
		行动力$M_c$         & 2    & 2      & 4 \bigstrut    \\
		\hline
		生命力上限$HP_c$    & 60   & 50     & 70 \bigstrut   \\
		\hline
		满血战斗力$F_c$     & 36   & 30     & 44 \bigstrut   \\
		\hline
	\end{tabular}%
	\label{兵团}%
\end{table}%

% Table generated by Excel2LaTeX from sheet 'Sheet1'
\begin{table}[htbp]
	\centering
	\caption{地形参数表}
	\begin{tabular}{c|c|c|c|c|}
		\hline
		地形                & 平原 & 山地 & 森林 & 沼泽 \bigstrut \\
		\hline
		地形战斗力增益$f_t$ & 0    & 5    & 3    & -3 \bigstrut   \\
		\hline
		地形行动力消耗$m_t$ & 2    & 4    & 3    & 4 \bigstrut    \\
		\hline
	\end{tabular}%
	\label{地形}%
\end{table}%



\end{document}